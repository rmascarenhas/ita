\documentclass[8pt,a4paper]{article}
\pagestyle{empty}
\usepackage{framed}
\usepackage{crlscode3e}

\begin{document}

\section*{Exercises}
My answers to a few selected exercises.

\begin{framed}
\textbf{\textit{2.1-3}} \\
\textit{Consider the \textbf{searching problem}: \\
\\
\textbf{Input}: A sequence of $n$ numbers $\langle a_1, a_2, ...., a_n \rangle$ and a value $v$. \\
\textbf{Output}: An index $i$ such that $v = A[i]$ or the special value \const{nil} if $v$
does not appear in $A$. \\
\\
Write pseudocode for \textbf{linear search}, which scans through the sequence, looking
for $v$. Using a loop invariant, prove that your algorithm is correct. Make sure
that your loop invariant fulfills the three necessary properties.}
\end{framed}

The algorithm:

\begin{codebox}
  \Procname{$\proc{Linear-Search}(A, v)$}
  \li $j = 1$
  \li \While $j < \attrib{A}{length} \const{and} A[j] \neq v$     \label{isort-loop-start}
        \Do
  \li     $j = j + 1$
  \End                                                            \label{isort-loop-end}
  \li \If $j \leq A.length$
        \Then
  \li     \Return $j$
  \li   \Else
  \li     \Return \const{nil}
        \End
\end{codebox}

The loop invariant of this algorithm:

\begin{quotation}
  At the start of each iteration of the \While loop of lines \ref{isort-loop-start}-\ref{isort-loop-end},
  the subarray $A[1 \twodots j]$ can only contain $v$ at position $j$.
\end{quotation}

As a consequence of the invariant above, we can infer that the subarray $A[1 \twodots j-1]$
does not contain $v$ for all iterations. Let's analyze the loop invariant properties
on the $\proc{Linear-Search}$ algorithm: \\

\textbf{Initialization}: Before the first loop iteration, $j = 1$, and hence $v$ can only
be at position $j$, since that is the only element in the array. \\

\textbf{Maintenance}: We see that the body of the \While loop only increments the $j$ index
until it gets out of the array bounds or the element at position $j$ equals $v$. On
every iteration of the loop, we know that the previous element is not equal to $v$. Thus,
the $v$ element can only be at position $j$, which was not compared yet. \\

\textbf{Termination}: The condition for the \While loop to terminate is if either
$j > \attrib{A}{length} = n$ or $A[j] = v$. In the former case, $j = \attrib{A}{length} + 1$
and the element $v$ cannot be in the array $A[1 \twodots n]$, that is, the array does
not contain $v$ and the algorith will return \const{nil}. In the latter case, $A[j] = v$,
and the value $v$ can only be at position $j$, which happens to be true. The algorithm
then returns the index $j$ where the value is located. Hence, the algorithm is correct.

\begin{framed}
\textbf{\textit{2.2-2}} \\
\textit{Consider sorting $n$ numbers stored in an array $A$ by first finding the smallest
element of $A$ and exchanging it with the element in $A[1]$. Then find the second smallest
element of $A$, and exchange it with $A[2]$. Continue in this manner for the first $n - 1$
elements of $A$. Write pseudocode for this algorithm, which is known as \textbf{selection sort}.
What loop invariant does this algorithm maintain? Why does it need to run for only the first
$n - 1$ elements, rather than for all $n$ elements? Give the best-case and worst-case running
times of selection sort in $\Theta$-notation.
}
\end{framed}

First of all, the algorithm:

\begin{codebox}
  \Procname{$\proc{Selection-Sort}(A)$}
  \li $j = 1$
  \li \For $j = 1 \To \attrib{A}{length} -1$  \label{ssort-loop-start}
        \Do
  \li     $min = j$
  \li     \For $i = j \To \attrib{A}{length}$
            \Do
  \li         \If $A[i] < A[min]$
                \Then
  \li             $min = i$
                \End
            \End
  \li     $key = A[j]$
  \li     $A[j] = A[min]$
  \li     $A[min] = key$
        \End                                  \label{ssort-loop-end}
\end{codebox}

The algorithm for $\proc{Selection-Sort}$ above holds the following loop invariant:

\begin{quotation}
  At the start of each iteration of the \For loop of lines \ref{ssort-loop-start}-\ref{ssort-loop-end},
  the subarray $A[1 \twodots j-1]$ is sorted and $A[j]$ is bigger than any element in it.
\end{quotation}

Let's take a look at the three properties of loop invariants: \\

\textbf{Initialization}: Before the first loop iteration, $j = 1$, and for trivial
reasons, $A[1 \twodots j-1]$ is sorted and $A[1]$ is bigger than any element in it. \\

\textbf{Maintenance}: In each iteration, the loop inserts the smallest element from
$A[j \twodots n]$ at position $j$, i.e., $A[j-1] \leq A[j]$ for any $j$. Thus, we conclude
that the subarray $A[1 \twodots j-1]$ is always sorted. As a consequence of the algorithm,
$A[j]$ cannot be smaller than any element in $A[1 \twodots j-1]$, otherwise that element
would already be there. \\

\textbf{Termination}: The loop terminates when $j = \attrib{A}{length}$. In this scenario,
the subarray in $A[1 \twodots j-1]$ is guaranteed to be ordered, and $A[j]$ is bigger than
any element in it. Hence, the whole array is sorted and the algorithm is correct. \\

As explained in the analysis above, the algorithm only needs to run for the first $n - 1$
elements because it guarantees that the next element considered is bigger than any element
before. Thus, at iteration $n$ the array will be already sorted.

As the algorithm loops through all the array no matter the disposition of the elements in
the array, this algorithm is $\Theta(n^{2})$ both in the best-case and worst-case scenarios.

\end{document}
