\documentclass[8pt,a4paper]{article}
\pagestyle{empty}
\usepackage{framed}
\usepackage{crlscode3e}

\begin{document}

\section*{Exercises}
My answers to a few selected exercises.

\begin{framed}
\textbf{\textit{2.1-3}} \\
\textit{Consider the \textbf{searching problem}: \\
\\
\textbf{Input}: A sequence of $n$ numbers $\langle a_1, a_2, ...., a_n \rangle$ and a value $v$. \\
\textbf{Output}: An index $i$ such that $v = A[i]$ or the special value \const{nil} if $v$
does not appear in $A$. \\
\\
Write pseudocode for \textbf{linear search}, which scans through the sequence, looking
for $v$. Using a loop invariant, prove that your algorithm is correct. Make sure
that your loop invariant fulfills the three necessary properties.}
\end{framed}

The algorithm:

\begin{codebox}
  \Procname{$\proc{Linear-Search}(A, v)$}
  \li $j = 1$
  \li \While $j < \attrib{A}{length} \const{and} A[j] \neq v$     \label{loop-start}
        \Do
  \li     $j = j + 1$
  \End                                                            \label{loop-end}
  \li \If $j \leq A.length$
        \Then
  \li     \Return $j$
  \li   \Else
  \li     \Return \const{nil}
        \End
\end{codebox}

The loop invariant of this algorithm:

\begin{quotation}
  At the start of each iteration of the \While loop of lines \ref{loop-start}-\ref{loop-end},
  the subarray $A[1 \twodots j]$ can only contain $v$ at position $j$.
\end{quotation}

As a consequence of the invariant above, we can infer that the subarray $A[1 \twodots j-1]$
does not contain $v$ for all iterations. Let's analyze the loop invariant properties
on the $\proc{Linear-Search}$ algorithm: \\

\textbf{Initialization}: Before the first loop iteration, $j = 1$, and hence $v$ can only
be at position $j$, since that is the only element in the array. \\

\textbf{Maintenance}: We see that the body of the \While loop only increments the $j$ index
until it gets out of the array bounds or the element at position $j$ equals $v$. On
every iteration of the loop, we know that the previous element is not equal to $v$. Thus,
the $v$ element can only be at position $j$, which was not compared yet. \\

\textbf{Termination}: The condition for the \While loop to terminate is if either
$j > \attrib{A}{length} = n$ or $A[j] = v$. In the former case, $j = \attrib{A}{length} + 1$
and the element $v$ cannot be in the array $A[1 \twodots n]$, that is, the array does
not contain $v$ and the algorith will return \const{nil}. In the latter case, $A[j] = v$,
and the value $v$ can only be at position $j$, which happens to be true. The algorithm
then returns the index $j$ where the value is located. Hence, the algorithm is correct.

\end{document}
