\documentclass[8pt,a4paper]{article}
\pagestyle{empty}
\usepackage{framed}
\usepackage{amsmath}

\begin{document}

\section*{Exercises}
My answers to a few selected exercises.

\begin{framed}
\textbf{\textit{3.1-2}} \\
\textit{Show that for any real constants $a$ and $b$, where $b > 0$, \
\\
\\
$(n + a)^{b} = \Theta(n^{b})$}
\end{framed}

We must find $c_{0}$, $c_{1}$ and $n_{0}$ such that

\begin{center}
  $c_{0}n^{b} \leq (n + a)^{b} \leq c_{1}n^{b}, \forall n \geq n_{0}$
\end{center}

An upper bound can be devised as:

\begin{equation*}
  \begin{split}
    n + a \leq n + |a| \leq 2n, \forall n \geq |a| \\
    \Rightarrow (n + a)^{b} \leq (2n)^{b} \\
    \Rightarrow (n + a)^{b} \leq 2^{b}n^{b}
  \end{split}
\end{equation*}

Similarly, a lower bound can be derived as:

\begin{equation*}
  \begin{split}
    n + a \geq n - |a| \geq \frac{n}{2}, \forall n \geq 2|a| \\
    \Rightarrow (n + a)^{b} \geq \left(\frac{n}{2}\right)^{b} \\
    \Rightarrow (n + a)^{b} \geq 2^{-b}n^{b}
  \end{split}
\end{equation*}

With the above results, we can establish tight bounds to the given function:

\begin{center}
  $2^{-b}n^{b} \leq (n + a)^{b} \leq 2^{b}n^{b}, \forall n \geq 2|a|$
\end{center}

which proves our hypothesis.

\begin{framed}
\textbf{\textit{3.1-4}} \\
\textit{Is $2^{n + 1} = O(2^{n})$? Is $2^{2n} = O(2^{n})$?}
\end{framed}

Our answers can be found if we expand both expressions:

\begin{center}
  $2^{n+1} = 2 \times 2^{n}$ \\
  $\Rightarrow 1 \times 2^{n} \leq 2^{n + 1} \leq 3 \times 2^{n}$
\end{center}

which proves that $2^{n + 1} = O(2^{n})$. On the other hand,

\begin{center}
  $2^{2n} = 2^{n} \times 2^{n}$
\end{center}

Let's suppose there is a constant $c$ such that:

\begin{center}
  $2^{2n} \leq c2^{n}$ \\
  $\Rightarrow 2^{n}2^{n} \leq c2^{n}$ \\
  $\Rightarrow 2^{n} \leq c$
\end{center}

Contradiction, since $c$ is a constant. Thus, $2^{2n} \neq O(2^{n})$.

\begin{framed}
\textbf{\textit{3.1-7}} \\
\textit{Prove that $o(g(n)) \cap \omega(g(n))$ is the empty set.}
\end{framed}

By the definitions of the $o$ and $w$ notations:

\begin{center}
  $f(n) \in o(g(n)) \iff f(n) < cg(n), \forall c > 0$ \\
  $f(n) \in \omega(g(n)) \iff f(n) > cg(n), \forall c > 0$
\end{center}

Thus, a function that is both $o(g(n))$ and $\omega(g(n))$ must be a
function such that

\begin{center}
  $cg(n) < f(n) < cg(n), \forall c > 0$
\end{center}

By definition, there cannot be such a function, proving that $o(g(n)) \cap \omega(g(n))$
is the empty set.

\begin{framed}
\textbf{\textit{3.2-2}} \\
\textit{Prove equation (3.16) \\
\begin{center}
  $a^{\log_{b}c} = c^{\log_{b}a}$
\end{center}
}
\end{framed}

We can prove the above equality by manipulating the left side of the equation and
using logarithmic rules:

\begin{equation*}
  \begin{split}
    \log_{c}(a^{\log_{b}c}) = \log_{b}c \times \log_{c}a = \\
    = \log_{b}c \times \left(\frac{\log_{b}a}{\log_{b}c}\right) = \log_{b}a \\
    \Rightarrow \log_{c}(a^{\log_{b}c}) = \log_{b}a \iff a^{\log{b}c} = c^{\log_{b}a}
  \end{split}
\end{equation*}

proving the equation (3.16).

\begin{framed}
\textbf{\textit{3.2-3}} \\
\textit{Prove equation (3.19). Also prove that $n! = \omega(2^{n})$ and $n! = o(n^{n})$ \\
\begin{center}
  $\lg(n!) = \Theta(n \lg n)$
\end{center}
}
\end{framed}

We can prove equation (3.19) by using Stirling's approximation:

\begin{equation*}
  \begin{split}
    \lg(n!) = \lg \left(\sqrt{2 \pi n} \left(\frac{n}{e}\right)^{n} \left(1 + \Theta \left(\frac{1}{n}\right)\right)\right) = \\
    = \lg \sqrt{2 \pi n} + n \lg \frac{n}{e} + \lg \left(1 + \Theta \left(\frac{1}{n}\right) \right) = \\
    = \frac{1}{2}\lg 2 \pi n + n \lg e^{-1}n + \lg (1 + \Theta(n^{-1}))
  \end{split}
\end{equation*}

Ignoring custom factors and expressing each of the functions in the sum above in terms of the $\Theta$-notation:

\begin{equation*}
  \lg(n!) = \Theta(\lg n) + \Theta(n \lg n) + \Theta(\lg n^{-1}) = \Theta(n \lg n)
\end{equation*}

which proves equation (3.19).

The other equations can be proven using their definitions:

\begin{equation*}
  \begin{split}
    \frac{n!}{n^{n}} = \frac{1}{n} \cdot \frac{2}{n} \cdot \frac{3}{n}
      \cdot ... \cdot \left(\frac{n - 1}{n}\right) \cdot \left(\frac{n}{n}\right)
      \Rightarrow \lim_{n \to \infty} \frac{n!}{n^{n}} = 0 \\
    \frac{n!}{2^{n}} = \frac{1}{2} \cdot \frac{2}{2} \cdot \frac{3}{2}
      \cdot ... \cdot \frac{n - 1}{2} \cdot \frac{n}{2}
      \Rightarrow \lim_{n \to \infty} \frac{n!}{2^{n}} = \infty
  \end{split}
\end{equation*}

\begin{framed}
\textbf{\textit{3.2-7}} \\
\textit{Prove by induction that the $i$ th Fibonacci number satisfies the equality
\begin{equation*}
  F_{i} = \frac{\phi^{i} - \widehat{\phi}^{i}}{\sqrt{5}}
\end{equation*}
where $\phi$ is the golden ratio and $\widehat{\phi}$ is its conjugate.
}
\end{framed}

  We start by proving the base cases, in which the Fibonacci function is not defined
in terms of itself:

\begin{equation*}
  \begin{split}
    F_{0} = \frac{\phi^{0} - \widehat{\phi}^{0}}{\sqrt{5}} = \frac{1 - 1}{\sqrt{5}} = 0 \\
    F_{1} = \frac{\phi^{1} - \widehat{\phi}^{1}}{\sqrt{5}} = 
      \frac{\left(\frac{1 + \sqrt{5}}{2}\right) - \left(\frac{1 - \sqrt{5}}{2}\right)}{\sqrt{5}} =
      \frac{\frac{2\sqrt{5}}{2}}{\sqrt{5}} = 1
  \end{split}
\end{equation*}

  Now, assuming that the hypothesis holds for $i - 1$, we must prove that it also holds for $i$:

\begin{equation*}
  \begin{split}
    F_{i} = F_{i - 1} + F_{i - 2}
      \Rightarrow F_{i} = \frac{\phi^{i - 1} - \widehat{\phi}^{i - 1}}{\sqrt{5}} + \frac{\phi^{i - 2} - \widehat{\phi}^{i - 2}}{\sqrt{5}} = \\
      \frac{\left(\phi^{i - 1} + \phi^{i - 2}\right) - \left(\widehat{\phi}^{i - 1} + \widehat{\phi}^{i - 2}\right)}{\sqrt{5}} =
      \frac{\phi^{i - 2}\left(\phi + 1\right) - \widehat{\phi}^{i - 2}\left(\widehat{\phi} + 1\right)}{\sqrt{5}} = \\
      \frac{\left(\phi^{i - 2} \cdot \phi^{2}\right) - \left(\widehat{\phi}^{i - 2} \cdot \widehat{\phi}^{2}\right)}{\sqrt{5}} =
      \frac{\phi^{i} - \widehat{\phi}^{i}}{\sqrt{5}}
  \end{split}
\end{equation*}

which proves our hypothesis.

\end{document}
